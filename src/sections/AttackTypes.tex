


\subsection{Compare and contrast types of attacks}
\begin{fullwidth}
    \begin{description}\itemsep2pt
        \item[Phishing]: Social engineering tactic to acquire personal information from a fake email with a clickable link.
        \item[Smishing]: (SMS phishing) The use of deceptive text messages into divulging sensitive information.
        \item[Vishing]: (Voice Phishing) Impersonates a trusted entity, such as a bank to trick into giving information.
        \item[Spam]: Unsolicited inappropriate messages sent with the purpose of spreading malware, advertising, phishing. 
        \item[Spear phishing]: Targeted type of phishing attack to make the scam convincing, often with insider information.
        \item[Dumpster diving]: Searching through an organization's or individual's trash to find sensitive information.
        \item[Shoulder surfing]: Observing a victim’s screen or keyboard to obtain sensitive information.
        \item[Pharming]: Manipulating the DNS system to resolve fake domain names, to lead them to a fake website.
        \item[Tailgating]: Physical security breach by following authorized person to get access to secure areas.
        \item[Eliciting information]: Psychological tactics to encourage individuals to share their knowledge willingly. 
        \item[Whaling]: Spear phishing for high-profile executives in an organization.
        \item[Prepending]: Organizing, manipulating, and structuring data in various applications.
        \item[Identity fraud]: An individual wrongfully obtains and uses someone else's personal data in a deceptive manner.
        \item[Invoice scams]: Impersonation of a legit business to deceive individuals into paying fraudulent invoices. 
        \item[Credential harvesting]: Tricking someone into disclosing login credentials to access sensitive info.
        \item[Reconnaissance]: Initial phase to gather intelligence via passive and active techniques.
        \item[Hoax]: Fabrication intended to deceive or trick individuals into believing false information or events.
        \item[Impersonation]: Masquerading as a legitimate user or entity to gain unauthorized access to information.
        \item[Watering hole attack]: Infecting a commonly visited website of a targeted specific group.
        \item[Typosquatting]: URL hijacking or domain squatting of fake domains which resemble a legit website.
        \item[Pretexting]: Fabricated scenario involving direct interaction to obtain sensitive information.
        \item[Influence campaigns]: Coordinated effort to shape public opinion, influence perceptions, and manipulate.
        \begin{itemize}\itemsep2pt
            \item[]\textbf{Hybrid warfare}: Blends conventional warfare tactics with unconventional method
        \end{itemize}
        \item[Principles] (reasons for effectiveness)
        \begin{itemize}\itemsep2pt
            \item \textbf{Authority}: The actor acts as an individual of authority
            \item \textbf{Intimidation}: Frightening or threatening the victim.
            \item \textbf{Consensus}: Convince based on what’s normally expected.
            \item \textbf{Scarcity}: Limited resources and time to act.
            \item \textbf{Familiarity}: The victim is well known.
            \item \textbf{Trust}: Gain their confidence, be their friend.
            \item \textbf{Urgency}: Limited time to act, rush the victim.
        \end{itemize}
    \end{description}
\end{fullwidth}
