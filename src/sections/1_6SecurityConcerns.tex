\begin{fullwidth} 
    \subsection{Explain the security concerns associated with various types of vulnerabilities}
    \subsubsection*{\color{red}Actors and Threats}
    \begin{description}
        \item[Cloud-based vs. On-premises Vulnerabilities:] Cloud-based vulnerabilities: Relate to the weaknesses within cloud services and platforms that can be exploited by attackers, such as misconfigured cloud storage or inadequate identity and access management. On-premises vulnerabilities: Concern issues in your own physical environment (like a server room in your building), like outdated firewalls or servers with unpatched software.
        \item[Zero-day:] A Zero-day vulnerability refers to a software security flaw that is known to the software vendor but doesn’t have a patch in place to fix the vulnerability. It’s called “zero-day” because the developers have “zero days” to fix the problem that has just been exposed — and perhaps already exploited by hackers.
    \end{description}
    \subsubsection*{\color{red}Weak Configurations}
    \begin{itemize}
        \item \textbf{Open Permissions:} Allowing too much access to too many people/users.
        \item \textbf{Unsecure Root Accounts:} Not protecting high-level administrative accounts properly.
        \item \textbf{Errors:} Mistakes in coding or system setup.
        \item \textbf{Weak Encryption:} Not using strong methods to protect data.
        \item \textbf{Unsecure Protocols:} Using outdated or insecure communication protocols.
        \item \textbf{Default Settings:} Not changing the settings that the system or application came with.
        \item \textbf{Open Ports and Services:} Leaving too many openings for attackers to potentially exploit.
    \end{itemize}

    \subsubsection*{\color{red}Third-party Risks}
    \begin{itemize}
        \item \textbf{Vendor Management:} Not properly overseeing or managing the organizations you buy products or services from.
        \item \textbf{System Integration:} Problems that might arise when trying to get different systems to work together.
        \item \textbf{Lack of Vendor Support:} Vendors not providing sufficient help or updates for their products.
        \item \textbf{Supply Chain:} The process of creating and delivering a product, which can be disrupted or exploited at various stages.
        \item \textbf{Outsourced Code Development:} Getting external parties to write software for you, which might not be as secure.
        \item \textbf{Data Storage:} Where and how you store data, and the vulnerabilities there.
    \end{itemize}

    \begin{description}
        \item[Legacy Platforms:] Using outdated systems or software that no longer receive updates and therefore, might be full of vulnerabilities.] 
    \end{description}

    \subsubsection*{\color{red}Improper or Weak Patch Management}
    \begin{itemize}
        \item \textbf{Firmware:} The foundational software for hardware, often neglected in the patching process.
        \item \textbf{Operating System (OS):} The main software that runs a computer, which might be left outdated.
        \item \textbf{Applications:} Programs used for various purposes that might not be kept up-to-date with security patches.
    \end{itemize}

    \subsubsection*{\color{red}Impacts}
    \begin{itemize}
        \item \textbf{Firmware:} The foundational software for hardware, often neglected in the patching process.
        \item \textbf{Data Loss:} Losing data due to an incident.
        \item \textbf{Data Breaches:} Unauthorized access to data.
        \item \textbf{Data Exfiltration:} The unauthorized copying, transfer, or retrieval of data.
        \item \textbf{Identity Theft:} Unauthorized use of someone’s personal data.
        \item \textbf{Financial:} Monetary losses from an incident.
        \item \textbf{Reputation:} Damage to the organization’s standing.
        \item \textbf{Availability Loss:} Losing access to systems, data, or networks.
    \end{itemize}
\end{fullwidth}
\newpage