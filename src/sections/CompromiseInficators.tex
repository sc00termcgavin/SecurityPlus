\subsection{Given a scenario, analyze indicators of compromise}
\subsubsection*{\color{green}Malware}
\begin{fullwidth}
    \begin{description}\itemsep2pt
        \item[Ransomware]: Denies access to a computer system or data until a ransom is paid.
        \item[Trojan]: A form of malware that pretends to be a harmless application.
        \item[Worm]: A self-contained infection that can spread itself through networks, emails, and messages.
        \item[PUP's]: Potentially Unwanted Programs software applications that may exhibit undesirable characteristics.
        \item[Memory-resident malware]: Operates primarily in a computer's volatile memory (RAM) rather than with files
        \item[Command and control]: (C2) Centralized server used by attackers to manager compromised devices.
        \item[Bots]: AI inside an infected machine performs specific actions as a part of a larger entity known as a botnet.
        \item[Cryptomalware]: A malicious program that encrypts programs and files on the computer to extort money.
        \item[Logic Bomb]: A malicious program that lies dormant until a specific date or event occurs.
        \item[Spyware]: Software that installs itself to spy and sends stolen info back to the host machine.
        \item[Keyloggers]: A malicious program that saves all of the keystrokes of the infected machine.
        \item[Remote Access Trojan]: (RAT) A remotely operated Trojan.
        \item[Rootkit]: A backdoor program that allows full remote access to a system.
        \item[Backdoor]: Allows for full access to a system remotely.
    \end{description}
\end{fullwidth}

\subsubsection*{\color{blue}Password Attacks}
\begin{fullwidth}
    \begin{description}
        \item[Sprying] a
    \end{description}
\end{fullwidth}

\subsubsection*{\color{red}Physical attacks}
\begin{fullwidth}
    \begin{description}
        \item[Sprying] a
    \end{description}
\end{fullwidth}