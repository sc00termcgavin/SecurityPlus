\begin{fullwidth} 
    \subsection{Given a scenario, analyze potential indicators associated with network attacks}
    \subsubsection*{\color{red}Wireless}
    \begin{description}
        \item[Evil Twin:] Imagine someone impersonating your Wi-Fi network to trick devices into connecting to it. It’s an “evil twin” of your legit Wi-Fi, stealing data and spying on users.
        \item[Rogue Access Point:] An unauthorized Wi-Fi access point, maybe added by an employee or attacker, which can bypass security settings.
        \item[Bluesnarfing:] Stealing information from Bluetooth-enabled devices by exploiting vulnerabilities in their Bluetooth connection.
        \item[Bluejacking:] Sending unsolicited messages to a Bluetooth device, mostly harmless but potentially annoying.
        \item[Disassociation:] Interrupting the Wi-Fi connection between a device and a network, causing disruptions.
        \item[Jamming:] Flooding a frequency (like Wi-Fi or cell frequencies) to block communications.
        \item[Radio frequency identification (RFID):] A tech that uses radio waves for tracking and identification but can be exploited to illicitly read information.
        \item[Near-field communication (NFC):] A way to wirelessly share data over short distances, like payment info, which can be exploited for unauthorized data access.
        \item[Initialization Vector (IV):] A random number used in cryptography for preventing predictability in encrypted data, but if not handled properly, can be a vulnerability. 
    \end{description}
    \subsubsection*{\color{Purple}Layer 2 Attacks}
    \begin{description}
        \item[ARP Poisoning:] Confusing network devices by sending fake Address Resolution Protocol messages, redirecting traffic through an attacker’s device.
        \item[Media access control (MAC) Flooding:] Overflowing the network switch with too many Media Access Control addresses, forcing it into acting like a basic hub and revealing internal data traffic.
        \item[MAC Cloning:] Copying a legit MAC address to impersonate a network device.
    \end{description}

    \subsubsection*{\color{blue}Domain Name System (DNS)}
    \begin{description}
        \item[Domain Hijacking:] Taking control of a domain away from the rightful owner, often for malicious activities.
        \item[DNS Poisoning:] Providing false DNS responses to redirect a user’s traffic to malicious sites.
        \item[URL Redirection:] Manipulating URLs to direct users to unintended pages, often for phishing.
        \item[Domain Reputation:] How trustworthy a domain is, based on its past actions and security posture.
    \end{description}

   
    \begin{description}
        \item[Distributed denial-of-service (DDoS):] Overwhelming a target, such as a website, with a flood of internet traffic, making it unavailable to users. Variants include targeting network, application, or operational technology layers.
        \item[On-Path Attack (Man-in-the-Middle):] This is like eavesdropping, where the attacker intercepts and possibly alters the communication between two parties without them knowing.
    \end{description}
    
    \subsubsection*{\color{green}Malicious code or script execution}
    \begin{description}
        \item[PowerShell, Python, Bash:] Different scripting languages that can be used to automate tasks or exploit vulnerabilities.
        \item[Macros, VBA:] Automated scripts, often in Office documents, that can be exploited to run malicious code.
    \end{description}
    
\end{fullwidth}

\newpage