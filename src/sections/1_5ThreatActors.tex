\begin{fullwidth} 
    \subsection{Different threat actors, vectors, and intelligence sources}
    \subsubsection*{\color{red}Actors and Threats}
    \begin{description}
        \item[dvanced Persistent Threat (APT):] Highly skilled attackers, often funded by governments, who aim to stealthily infiltrate and stay in networks for a long time, usually for espionage.
        \item[Insider Threats:] People inside an organization (like employees or contractors) who pose security risks, either maliciously or inadvertently.
        \item[State Actors:] Hackers sponsored by national governments to engage in cyber espionage, warfare, or sabotage.
        \item[Hacktivists:] Individuals or groups hacking for political or social reasons rather than financial gain.
        \item[Script Kiddies:] Inexperienced hackers who use pre-written scripts or tools to perform attacks, without much understanding of how they work.
        \item[Criminal Syndicates:] Organized crime groups engaging in cybercrime for financial gain.
        \item[Hackers:] People who find and exploit vulnerabilities in systems. They can be:
        \begin{itemize}
            \item \textbf{Authorized:} Have permission to access.
            \item \textbf{Unauthorized:} No permission to access.
            \item \textbf{Semi-authorized:} Somewhere in between; maybe they had permission at one point or for certain tasks.
        \end{itemize}
        \item[Shadow IT:] Unauthorized tech solutions used inside an organization without the IT department’s knowledge or approval.
        \item[Competitors:] Business rivals who might engage in cyber tactics to gain a competitive edge.
    \end{description}

    \subsubsection*{\color{red}Attributes of Actors}
    \begin{itemize}
        \item \textbf{Internal/External:} Are they inside or outside the organization?
        \item \textbf{Level of Sophistication:} How skilled are they?
        \item \textbf{Resources:} What tools, money, or people do they have at their disposal?
        \item \textbf{Intent/Motivation:} Why are they doing what they’re doing?
    \end{itemize}

    \subsubsection*{\color{purple}Vectors}
    \begin{itemize}
        \item \textbf{Direct Access:} Physically accessing systems.
        \item \textbf{Wireless:} Via Wi-Fi, Bluetooth, etc.
        \item \textbf{Email:} Think phishing or malware attachments.
        \item \textbf{Supply Chain:} Targeting suppliers or service providers.
        \item \textbf{Social Media:} Spreading malware or misinformation.
        \item \textbf{Removable Media:} USB drives, DVDs, etc.
        \item \textbf{Cloud:} Exploiting vulnerabilities in cloud services.
    \end{itemize}

    \subsubsection*{\color{purple}Threat Intelligence Sources}
    \begin{itemize}
        \item \textbf{Open-Source Intelligence (OSINT):} Publicly available info.
        \item \textbf{Vulnerability Databases:} Listings of known security vulnerabilities.
        \item \textbf{Public/Private Information-Sharing Centers:} Organizations that share threat data.
        \item \textbf{Dark Web:} A part of the internet not indexed by search engines, often hosting illegal activities.
        \item \textbf{Indicators of Compromise:} Signs that a breach has occurred.
        \item \textbf{Automated Indicator Sharing (AIS), STIX/TAXII:} Tools and formats for sharing threat intelligence.
        \item \textbf{Predictive Analysis:} Forecasting future threats.
        \item \textbf{Threat Maps:} Visual representation of ongoing cyber attacks globally.
        \item \textbf{File/Code Repositories:} Places where software code is stored, which can sometimes contain vulnerabilities.
    \end{itemize}

    \subsubsection*{\color{purple}Research Sources}
    \begin{itemize}
        \item \textbf{Vendor Websites:} Companies that make software/hardware often provide updates or alerts.
        \item \textbf{Conferences:} Where experts discuss the latest in cybersecurity.
        \item \textbf{Academic Journals:} Peer-reviewed publications on new findings.
        \item \textbf{Request for Comments (RFC):} Official documentations and standards.
        \item \textbf{Local Industry Groups:} Local or regional groups focusing on security.
        \item \textbf{Social Media:} Real-time info, but needs verification.
        \item \textbf{Threat Feeds:} Live data streams about potential threats.
        \item \textbf{Adversary Tactics, Techniques, and Procedures (TTP):} Documented strategies used by attackers.
    \end{itemize}    
\end{fullwidth}
\newpage


